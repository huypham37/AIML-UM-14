\section{Algorithm Performance Analysis}

\subsection{Learning Algorithm Parameters}

% Impact of learning rates
\subsubsection{Learning Rates}
Experiments were conducted with varying learning rates to determine their impact on training performance.

\begin{figure}[h]
    \centering
    \includegraphics[width=0.8\textwidth]{images/learning_rate_comparison.png}
    \caption{Training reward vs. steps for different learning rates.}
    \label{fig:learning_rate}
\end{figure}

% Neural network architecture variations
\subsubsection{Neural Network Architectures}
The effect of different neural network architectures was analyzed.

\begin{table}[h]
    \centering
    \caption{Performance comparison of different neural network architectures.}
    \begin{tabular}{lccc}
        \toprule
        Architecture & Layers & Hidden Units & Training Time (s) \\
        \midrule
        Small & 2 & 64 & 1200 \\
        Medium & 3 & 128 & 1800 \\
        Large & 4 & 256 & 2500 \\
        \bottomrule
    \end{tabular}
    \label{tab:nn_architectures}
\end{table}

% Environment copies analysis
\subsubsection{Environment Copies}
Training with multiple environment copies was investigated to determine its impact on training speed and resource utilization.

% Training duration and convergence rates
\subsubsection{Training Duration and Convergence}
The relationship between training duration and convergence rates was analyzed for each algorithm.

\subsection{Input/Sensor Configuration Analysis}

% Comparison of different sensor configurations
\subsubsection{Sensor Configurations}
Different sensor configurations (e.g., raycasts, cameras, custom sensors) were compared.

% Impact on training efficiency
\subsubsection{Impact on Training Efficiency}
The effect of sensor configuration on training efficiency was evaluated.

% Memory and processing overhead
\subsubsection{Memory and Processing Overhead}
The memory and processing overhead associated with each sensor type were measured.

% Evaluation of custom sensor implementation
\subsubsection{Custom Sensor Implementation}
[If you implemented a custom sensor, describe its performance characteristics here.]

\subsection{Training Environment Analysis}

% In-editor vs compiled executable performance
\subsubsection{In-Editor vs. Compiled Executable}
Performance was compared between training within the Unity editor and training with a compiled executable.

% Scaling behavior with environment complexity
\subsubsection{Scaling with Environment Complexity}
Experiments were conducted to analyze how performance scales with increasing environment complexity.

% Resource utilization patterns
\subsubsection{Resource Utilization Patterns}
Resource utilization patterns (CPU, GPU, memory) were analyzed for different environment complexities.

