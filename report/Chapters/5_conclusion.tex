\section{Conclusions}

% Summary of key findings
\subsection{Summary of Findings}
This project analyzed the performance of ML-Agents in the Unity engine. The key findings include:
\begin{itemize}
    \item [Finding 1, e.g., Learning rate has a significant impact on convergence speed.]
    \item [Finding 2, e.g., GPU utilization is often the primary bottleneck during training.]
    \item [Finding 3, e.g., Increasing environment complexity leads to a linear increase in training time.]
    \item [Finding 4, e.g., Training with a compiled executable is significantly faster than training in the editor]
    \item ...
\end{itemize}

% Recommendations for optimal configuration
\subsection{Recommendations}
Based on the results, the following recommendations are made for optimal training performance:
\begin{itemize}
    \item Use a learning rate between [range].
    \item Use a batch size of [value].
    \item Train with [number] environment copies if GPU memory allows.
    \item Prefer [sensor type] for better efficiency.
    \item Consider a neural network architecture with [number] layers and [number] hidden units.
    \item Compile the environment to an executable for faster training.
\end{itemize}

% Lessons learned and best practices
\subsection{Lessons Learned and Best Practices}
\begin{itemize}
    \item \textbf{Profiling is Crucial:} Regularly profile training sessions to identify bottlenecks.
    \item \textbf{Iterative Optimization:} Optimize parameters iteratively, starting with the most impactful ones.
    \item \textbf{Environment Design:} Carefully design environments to balance complexity and performance.
    \item \textbf{Hardware Considerations:} Choose appropriate hardware (especially GPU) based on the complexity of the task.
    \item \textbf{Leverage ML-Agents Features:} Utilize features like multi-environment training and curriculum learning for faster convergence.
\end{itemize}