\section{Methodology}

\subsection{Experimental Setup}

% Hardware specifications
\subsubsection{Hardware}
\begin{itemize}
    \item CPU: [e.g., Intel Core i7-12700K]
    \item GPU: [e.g., NVIDIA GeForce RTX 3080]
    \item RAM: [e.g., 32 GB DDR5]
    \item Operating System: [e.g., Windows 10]
\end{itemize}

% Software versions and configurations
\subsubsection{Software}
\begin{itemize}
    \item Unity Version: [e.g., 2022.3.0f1]
    \item ML-Agents Release: [e.g., Release 20]
    \item Python Version: [e.g., 3.9]
    \item TensorFlow/PyTorch Version: [e.g., TensorFlow 2.10.0]
    \item CUDA/cuDNN Version: [e.g., CUDA 11.7, cuDNN 8.5]
\end{itemize}

% Testing environments selected
\subsubsection{Testing Environments}
\begin{itemize}
    \item Environment 1: [Detailed description of environment setup, parameters, etc.]
    \item Environment 2: [Detailed description]
    \item ...
\end{itemize}

% Metrics tracked (wall time, CPU usage, GPU usage, memory, framerate)
\subsubsection{Metrics Tracked}
The following performance metrics were collected:
\begin{itemize}
    \item \textbf{Wall Time:} Total time taken for training sessions.
    \item \textbf{CPU Usage:} Percentage of CPU utilization.
    \item \textbf{GPU Usage:} Percentage of GPU utilization, including memory usage.
    \item \textbf{Memory Usage:} System RAM usage during training.
    \item \textbf{Framerate:} Frames per second (FPS) in Unity during training (when applicable).
\end{itemize}

\subsection{Performance Metrics}

% Description of how each metric was measured
\subsubsection{Measurement Methodology}
\begin{itemize}
    \item \textbf{Wall Time:} Measured using Python's `time` module.
    \item \textbf{CPU Usage:} Collected using [e.g., the `psutil` library in Python].
    \item \textbf{GPU Usage:} Collected using [e.g., `nvidia-smi` and the `pynvml` library].
    \item \textbf{Memory Usage:} Measured using [e.g., the `psutil` library].
    \item \textbf{Framerate:} Measured using Unity's built-in Stats window or Profiler.
\end{itemize}

% Unity Profiler setup and configuration
\subsubsection{Unity Profiler Configuration}
The Unity Profiler was used to collect detailed performance data within the Unity Editor. The following settings were used:
\begin{itemize}
    \item Profiler Mode: [e.g., Deep Profile]
    \item Modules Enabled: [e.g., CPU, GPU, Memory, Physics]
    \item ...
\end{itemize}

% Data collection methods
\subsubsection{Data Collection Methods}
Data was collected at regular intervals during training sessions. For each metric, the average, minimum, and maximum values were recorded.

% Tools and scripts used for measurements
\subsubsection{Tools and Scripts}
\begin{itemize}
    \item \textbf{Unity Profiler:} For in-editor performance analysis.
    \item \textbf{Python Scripts:} Custom scripts using libraries like `psutil`, `pynvml`, and `time` for collecting system-level metrics.
    \item \textbf{TensorBoard:} For visualizing training progress and performance metrics.
    \item [List any other tools like `nvidia-smi`, ML-Agents Profiler, etc.]
\end{itemize}

\begin{lstlisting}[language=Python, caption=Example Python Script for Monitoring]
import psutil
import time

while True:
    cpu_percent = psutil.cpu_percent()
    mem_percent = psutil.virtual_memory().percent
    print(f"CPU: {cpu_percent}%, Memory: {mem_percent}%")
    time.sleep(1)
\end{lstlisting}